\subsubsection{EI}
Expected improvement is an acquisition function frequently used in Bayesian optimization used to quantify how likely it is that an improvement over the current known optimum is to be found at a different location, according to the surrogate model. Technically this involves quantifying the probability mass of the surrogate model above (or below) the known optimum at a given point.

In essence this is what we are trying to do with SQ: to quantify expected improvement of one distribution related to another distribution. However, there are some non-trivial differences between what we want and EI though. EI compares the surrogate model at a given point, to the current optimum estimate (not a distribution). Instead we would like to calculate the overlap of probability mass between two distributions.

If the trusted has more positive probability mass than the candidate then the SQ should be higher, and vice versa.

We can consider two arbitrary distributions with sufficient statistics $\mu$ and $\sigma$. We can approximate these distributions via Normal distributions with the same sufficient statistics. Calculating the probability overlap between the two involves calculating the intersection points of the pdf equations, and then integrating. Sometimes there can be multiple intersections, and this complicates things.

While trying to understand the best way to do this, I found the ``Hellinger Metric'' which is a distance measure between two distributions, where $0$ signifies total overlap of probability, and $1$ signifies total separation of probability.

We decided to pursue using the Hellinger metric for SQ.
