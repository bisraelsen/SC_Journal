\section{Conclusions} \label{sec:conclusions}
Unmanned autonomous physical systems are able to tackle complex decision making problems for high-consequence applications, but in order to be able to reduce the amount of supervision required these systems need to be able to perform self-assessment, or introspection. We draw on \emph{Factorized Machine Self-Confidence (\famsec)} which is a framework of self-assessments that enable an APS to quantify its own capabilities.

The `Donut Delivery' problem was introduced, and an Amazon Mechanical Turk experiment was designed to study 

The simulations run so far have not directly considered `different classes' of solvers, however as \xQ{} only depends on reward distributions \rwd{}, and \rwdstar{} this is not a limitation. Also, since the calculation of \xQ{} generally depends on \rwdstar{} predicted from \surrogate{} it would be prudent to enable the surrogate to predict \rwdstar{} as well as an associated uncertainty in order to have an indication of where \surrogate{} can be trusted.

Another direction for future work is to develop approaches for the remaining three \famsec{} factors. Each of the individual factors reflects a critical meta-assessment of the competency of the APS.

\brett{future work with a more `relevant task', that is more complex. Future work to understand how to find the right ranges\ldots we just did a simple mapping to words, combining factors into a single factor (perhaps taking the minimum of the two)}