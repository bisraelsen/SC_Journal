\section{Donut Delivery Experiment: Training and Comprehension}
\subsection{Training}
Participants were trained prior to proceeding to the task set. The following general training was given to all participants:

\begin{quoting}
    Donut Delivery Dispatch

    Welcome! You are the dispatch supervisor for a fleet of autonomous delivery trucks. These trucks deliver donuts to several businesses in cities all over the state. As dispatch supervisor, you will be deciding which donut orders you accept for delivery, as well as those that you need to decline.

    This might sound simple, but there is a complication you will need to consider. There is an active ring of motorcycle-riding donut bandits in the state. They seem to have it out for your donut delivery business. The bandits don't know where you are, or where you're heading, but if they cross your path, they steal your customer's donuts! You'll have many angry customers as a result.

    The autonomous trucks are able to make deliveries without any human help. They are able to consider the city map, the road conditions, weather, and traffic. The trucks use this information to plan a route to the destination while also avoiding the gang.

    Each time an order is placed, you have the responsibility to decide whether or not the truck can succeed in delivering the donuts. In order to keep your customers happy, you must be confident that the truck can deliver the donuts as promised. If you believe it is unlikely for the donuts to reach their intended destination unharmed, you should not accept the order. However, you'll lose potential customers if you continually decline requests for donut delivery.
\end{quoting}

After the introduction to the task, participants are then trained further.

\begin{quoting}
    When a donut order is placed for delivery, you will see something like the image shown below. This is a map of a city; the black dots represent intersections, and lines are roads that connect them (if there are overlapping lines, these indicate a tunnel or a bridge).

    \begin{figure}[h]
        \centering
        \includegraphics[width=0.4\linewidth]{Figures/transition_vary_1_net_4}
    \end{figure}

    \begin{itemize}
        \item The delivery truck's current position (the green dot with a truck).
        \item The last known location of the donut bandits (the red dot with a motorcycle).
        \item The delivery destination (the black star).
    \end{itemize}

    After you assess the situation you must choose whether to 'accept' or 'decline' the delivery. You do this by pressing the up key to accept, or the down key to decline. If you choose to accept the delivery you will see the outcome of the delivery (whether it was successful or not). The score breakdown is as follows:

    \begin{itemize}
        \item Successful delivery: {\color{blue}+1 point}
        \item Failed delivery: {\color{blue}-1 point}
        \item Declined delivery: {\color{blue}-1/4 point}
    \end{itemize}

    You will repeat this process for approximately 30 deliveries. At the end of this HIT you will paid \textbf{a bonus of up to \$1.00} based on your final score, and the effort you put in to the experiment (please don't push the same key the whole time, or randomly select responses without thinking).
\end{quoting}

Those in conditions with \xQ{} `present' were further trained as follows:

\begin{quoting}
    Solver Quality represents the truck's assessment of its ability to make a plan, which is based on previous plan making in similar situations. Solver Quality can be: `very bad', `bad', `okay', `good', or `very good'. A `bad' Solver Quality indicates that the truck does not think it will make a good plan based on previous experience. A `good' Solver Quality indicates that the truck thinks it will be able to make a good plan. A `very good' Solver Quality indicates that the truck is extremely confident in the plan it can make. For example:

    \begin{itemize}
        \item In a city whose road layout is much more complex than has been encountered before, Solver Quality will likely be `bad'
        \item When the destination is near to the truck, Solver Quality will likely be `good'
        \item In a well-mapped small city with little traffic, the truck would likely report a `very good' Solver Quality
    \end{itemize}

    Solver Quality \textbf{does not} indicate whether the delivery will be a success or a failure, but only the capability of making the best possible plan. Even the best plans can fail in a losing situation.

\end{quoting}

Those in conditions with \xP{} `present' were further trained as follows:

\begin{quoting}
    Outcome Assessment represents the truck's assessment of the possible outcomes of a delivery. It can be: `very bad', `bad', `uncertain', `good', or `very good'. When the truck reports a `very bad' Outcome Assessment that indicates that the truck is nearly certain of a bad outcome (failure), a `very good' Outcome Assessment means the truck is nearly certain of a good outcome (success). An `uncertain' Outcome Assessment indicates that the truck thinks the chances of success or failure are nearly equal. For example:

    \begin{itemize}
        \item If there is only one path to take to the customer, and the motorcycle gang is on that path, the Outcome Assessment will likely be `very bad'
        \item If the city is a grid, that is easy to navigate, and it will also be easy to avoid the gang, then the truck's Outcome Assessment will likely be `very good'
        \item If success/failure of a delivery depends on a single event going well (i.e. being able to make a certain light), then the truck will likely report an `uncertain' Outcome Assessment
    \end{itemize}

    Outcome Assessment \textbf{does not indicate} the truck's ability to make a plan, only what it thinks the outcome of a given plan will be. The Outcome Assessment of a bad plan is not accurate.
\end{quoting}

\subsection{Ensuring Understanding}
In order to ensure participants correctly understood the task they were required to pass quizzes on key concepts of the task. The correct responses are indicated with boldface. For all conditions participants were given two chances to answer all questions correctly before proceeding, they were allowed to re-read the instructions if necessary. If they were unable to correctly respond to the questions, they could not proceed further and were not compensated.

\paragraph{General Questions} The following general questions were asked to ensure that participants understood the overall goals of the task. 
\begin{quoting}
\begin{enumerate}
    \item Which of the following *most completely* describes your duties as a dispatch supervisor?
    \begin{enumerate}[label=(\alph*)]
        \item Send customers their donuts!
        \item \textbf{Decide whether to make deliveries after considering relevant information.}
        \item Wage a long-term battle with the motorcycle gang.
        \item None of the above.
    \end{enumerate}
    \item Which of the following factors is *least likely* to affect the successful delivery of donuts?
    \begin{enumerate}[label=(\alph*)]
        \item The likelihood of running into the motorcycle gang
        \item The quality of the roads
        \item How far away the customer is from the delivery truck
        \item \textbf{How many donuts the customer wants}
    \end{enumerate}
    \item On the map, what does the black star represent?
    \begin{enumerate}[label=(\alph*)]
        \item Delivery Truck
        \item Last reported location of Motorcycle Gang
        \item \textbf{Destination for the delivery}
        \item Point of interest for visitors to the city
    \end{enumerate}
    \item On the map, what does the green dot with a truck represent?
    \begin{enumerate}[label=(\alph*)]
        \item \textbf{Delivery Truck}
        \item Last reported location of Motorcycle Gang
        \item Destination for the delivery
        \item Point of interest for visitors to the city
    \end{enumerate}
    \item On the map, what does the red dot with a motorcycle represent?
    \begin{enumerate}[label=(\alph*)]
        \item Delivery Truck
        \item \textbf{Last reported location of Motorcycle Gang}
        \item Destination for the delivery
        \item Point of interest for visitors to the city
    \end{enumerate}
    \item Which of the following contributes to the difficulty of your job?
    \begin{enumerate}[label=(\alph*)]
        \item You're inside and the lights are too dim
        \item You don't like donuts, especially the cake ones
        \item The trucks are old
        \item \textbf{You don't have direct access to all relevant information}
    \end{enumerate}
\end{enumerate}
\end{quoting}

\paragraph{\xQ{} `present' condition} The following questions were only asked to participants in the \xQ{} `present' condition:
\begin{quoting}
\begin{enumerate}
    \item From the truck’s observations, it appears that the built-in map is out of date
    \begin{enumerate}[label=(\alph*)]
        \item \textbf{Bad}
        \item Good
    \end{enumerate}
    \item The destination is pretty close to the truck, and the maps are accurate
    \begin{enumerate}[label=(\alph*)]
        \item Bad
        \item \textbf{Good}
    \end{enumerate}
\end{enumerate}
\end{quoting}

\paragraph{\xO{} `present' condition} The following questions were only asked to participants in the \xO{} `present' condition:
\begin{quoting}
\begin{enumerate}
    \item The success of the delivery depends heavily on the traffic at a single intersection.
    \begin{enumerate}[label=(\alph*)]
        \item Bad
        \item \textbf{Uncertain}
        \item Good
    \end{enumerate}
    \item Given the truck’s location with respect to that of the Motorcycle gang, it will be difficult to avoid the gang.
    \begin{enumerate}[label=(\alph*)]
        \item \textbf{Bad}
        \item Uncertain
        \item Good
    \end{enumerate}
\end{enumerate}
\end{quoting}

When participants successfully completed the applicable quizzes they then moved on to the experimental task set. The task set was a set of 43 different road-network problems, similar to the one shown in the instructions above. After completing the tasks participants saw the following information:

\begin{quoting}
    Good Job. We'll Review Your Results, And Calculate Your Bonus

    Please Wait While We Load The Final Survey
\end{quoting}

Participants were never shown their cumulative score (both during the task set and after completion) because this may have influenced their survey responses. Having said that, they were shown the points earned from each individual choice. We wanted responses to be based on their perceptions of the interaction with the autonomous system, and not be influenced by a fairly arbitrary score point total. This is especially important since users had no prior experience to suggest was a `good' or `bad' score on the task was (e.g. `I got 10 points, but that doesn't seem very good', when in actuality nobody scored a 10).

\subsection{Survey Questions}
The following were the questions asked to all participants of the study. Except as noted, all responses were on a Likert scale from 0 to 4. Labels for the Likert scales of different questions are indicated.

\brett{\textbr{At some point we should cite the experiment repository}}

\begin{quoting}
\begin{enumerate}[label=\textbf{Survey~\arabic*)}]
    \item To what extent was the delivery truck able to make deliveries? (scale: Likert, range: [`Very Incapable', `Very Capable'])
    \item To what extent do you feel you were able to predict the delivery truck's performance from delivery to delivery? (range [`Very Incapable', `Very Capable'])
    \item To what extent could you rely on the delivery truck to perform its job? (range: [`Not at all', `Relied Heavily'])
    \item To what extent did the delivery truck perform similarly in related tasks? (range: [`Very inconsistently', `Very consistently'])
    \item To what extent do you agree with the following: The delivery truck is well designed. (range: [`Very poorly designed', `Very well designed'])
    \item To what extent do you agree with the following: In the future I would like to use this delivery truck to fulfill deliveries. (range: [`It causes trouble', `It's necessary'])
    \item In future interaction, to what extent would you trust the delivery truck? (range: [`Not at all!', `A Lot!'])
    \item To what extent did you find the city maps useful? (range: [`Not at all!', `A Lot!'])
\end{enumerate}
\end{quoting}

Two specific questions were asked about each of the \famsec{} metrics that were displayed (if any were displayed at all).
\begin{quoting}
\begin{enumerate}[label=\textbf{\famsec~\arabic*)}]
    \item To what extent did the [\emph{metric name}] metric assist you in making decisions? (range: [`Played no role', `Played a large role']) 
    \item The [\emph{metric name}] metric was easy to understand. (range: [`Strongly Disagree', `Strongly Agree']) 
\end{enumerate}
\end{quoting}

The following demographic questions were asked in order to ensure that conditions had reasonably similar participants.
\begin{quoting}
\begin{enumerate}[label=\textbf{Demo~\arabic*)}]
    \item What age range do you fall within? (non-Likert choices: [18-25, 25-30, 31-35, 36-40, 41-45, 46-50, 50+])
    \item To what extent are you comfortable using technology? (range: [`Avoid it', `Second nature'])
    \item To what extent do you feel you personally rely on technology on a daily basis? (range: [`Unnecessary', `Necessary'])
    \item What is your gender? (non-Likert choices: [`I would rather not say', `Female', `Male'])
\end{enumerate}
\end{quoting}

Participants were also given a chance to share opinions via an unstructured text entry.

