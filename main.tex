\documentclass[conference,10pt]{IEEEtran}
\usepackage{times}

\usepackage[numbers]{natbib}
\usepackage{multicol}
\usepackage{booktabs}
\usepackage{tabulary} %for text tables
\usepackage{graphicx,caption}
\usepackage{mathtools}%loads amsmath
\usepackage{bm}
\usepackage{amssymb,amsfonts}
\usepackage{subcaption}
% \usepackage[caption=false]{subfig}
% \captionsetup{width=\linewidth}
\usepackage{enumitem}
\usepackage[bookmarks=true]{hyperref}
\usepackage{xcolor}
% \usepackage{etoolbox}

\usepackage[indentfirst=false,leftmargin=1cm,font=itshape,font+=small]{quoting}

% \newenvironment{myquote}[1]%
    % {\setlength{\leftmargini}{#1}\color{blue}\fontfamily{droid}{\quotation}{\endquotation}}
    % {\list{}{\leftmargin=#1\rightmargin=#1}\item[]}%
    % {\endlist}

\AtBeginEnvironment{quote}{\fontfamily{droid}\small}

%%%%% shortcut commands
\newcommand{\famsec}{FaMSeC}
\newcommand{\solve}{$\mathcal{S}$}
\newcommand{\solvestar}{$\mathcal{S}^*$}
\newcommand{\taskclass}{$c_{\mathcal{T}}$}
\newcommand{\task}{$\mathcal{T}$}
\newcommand{\taski}{$\mathcal{T}_i$}
\newcommand{\taskN}{$\mathcal{T}_N$}
\newcommand{\rwd}{$\mathcal{R}$}
\newcommand{\rwdstar}{$\mathcal{R}^*$}
\newcommand{\rwdstari}{$\mathcal{R}^*_i$}
\newcommand{\rwdstarN}{$\mathcal{R}^*_N$}
\newcommand{\rwdstarapprox}{$\widetilde{\mathcal{R}}^*$}
\newcommand{\rwdstariapprox}{$\widetilde{\mathcal{R}}^*_i$}
\newcommand{\policy}{$\mathcal{\pi}$}
\newcommand{\policystar}{$\mathcal{\pi}^*$}
\newcommand{\surrogate}{$\mathcal{M}^*(\mathcal{T})$}
\newcommand{\xQ}{$x_Q$} %solver quality
\newcommand{\xO}{$x_O$} %outcome assessment
\newcommand{\xP}{$x_P$} %past performance
\newcommand{\xI}{$x_I$} %command interpretation
\newcommand{\xM}{$x_M$} %model validity
\newcommand{\xSC}{$x_{SC}$=\{\xI,\xM,\xQ,\xO,\xP \}}
\newcommand{\dkl}{$D_{KL}$}
\newcommand{\hell}{$H^2$}
\newcommand{\hellmet}{\hell(P,Q)}
\newcommand{\dmu}{\Delta \mu}
\def\-{\raisebox{.75pt}{-}} %short negative sign
%%%%%%%%%%
\newcommand{\hlr}[1]{{\color{red} #1}}
\newcommand{\hlb}[1]{{\color{blue} #1}}
\newcommand{\hlo}[1]{{\color{orange} #1}}
\newcommand{\nisar}[1]{\hlr{NRA: #1}}
\newcommand{\brett}[1]{\hlb{BWI: #1}}

%%%%%%%%%%

\pdfinfo{
    /Author (Brett Israelsen)
    /Title (Something...)
    /CreationDate (D:201812201111)
    /Subject (Human-Robot Trust)
    /Keywords (Algorithmic Assurance, Trust, Self-Confidence)
}

\begin{document}

\title{The Effects of Machine Self-Confidence on User Behavior and Perception---\brett{Working Title}}
\author{Author Names Ommitted for Anonymous Review. Paper-ID [add ID here]} 

\maketitle

\begin{abstract}
    \brett{Revisit this}
    It is difficult for users to comprehend the capabilities and limitations of advanced robotic systems. In some cases, these limitations are not even known or quantified. Designers and users who aren't aware of the limitations of the autonomous robot they are using are likely to use it inappropriately. Machine Self-Confidence is the ability for a machine to quantify its competency boundaries; once known they can be communicated to interested humans (users or designers). Herein, two assessments of robot capability are presented: `Outcome Assessment' and `Solver Quality'. Results from an Amazon Mechanical Turk experiment meant to investigate their effects on user behavior and perception indicate that the communication of both assessments to human users has a significant impact on behavior and perception.
\end{abstract}

\IEEEpeerreviewmaketitle

\input{"introduction.tex"}
\input{"FaMSeC.tex"}
\input{"methodology.tex"}
\input{"results.tex"}
\input{"conclusions.tex"}

\bibliographystyle{plainnat}
\bibliography{References}

\appendices
\input{"appendix.tex"}

\end{document}
